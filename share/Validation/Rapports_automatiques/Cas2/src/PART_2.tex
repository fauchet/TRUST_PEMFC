
%\section{Transport d'esp�ces Darcy / Knudsen}


\subsection{Int�r�t du cas-test}

Le principal int�r�t de ce cas est de tester la capacit� traiter un
probl�me de transport coupl� de plusieurs esp�ces prenant en compte
l'effet de la couche de Knudsen et le transport de type Darcy.

\begin{center}
\input{\orig/transp_Darcy_Kn.pdftex_t}
\par\end{center}


\subsection{G�om�trie}

La hauteur et la longueur du domaine sont respectivement:
\[
h_{GDL}=200\text{ \textmu m}
\]
\[
L=2\text{ mm}
\]



\subsection{�quations}

Pour simplifier la lecture du syst�me d'�quations suivants, on adopte
le code couleur suivant:
\begin{itemize}
\item en \textcolor{red}{rouge} les variables principales;
\item en \textcolor{magenta}{rose} les variables interm�diaires, c'est-�-dire
les variables qui s'expriment � partir des variables principales;
\item en \textcolor{cyan}{bleu clair} les constantes.
\end{itemize}
\[
{\color{cyan}\epsilon}\,\frac{\partial{\color{red}c_{O_{2}}}}{\partial t}=-\nabla\cdot{\color{magenta}\boldsymbol{N}_{O_{2}}}
\]


\[
{\color{cyan}\epsilon}\,\frac{\partial{\color{red}c_{vap}}}{\partial t}=-\nabla\cdot{\color{magenta}\boldsymbol{N}_{vap}}
\]

\[
{\color{cyan}\epsilon}\,\frac{\partial{\color{red}c_{N_{2}}}}{\partial t}=-\nabla\cdot{\color{magenta}\boldsymbol{N}_{N_{2}}}
\]

Les flux des esp�ces $\boldsymbol{N}_{i}$ sont calcul�s � partir
des gradients de concentration et du gradient de pression en inversant
un syst�me d'�quations.
\[
\begin{bmatrix}\left(-\frac{{\color{magenta}X_{N_{2}}}}{{\color{cyan}\left(D_{A}\right)_{O_{2},N_{2}}}}-\frac{{\color{magenta}X_{vap}}}{{\color{cyan}\left(D_{A}\right)_{O_{2},vap}}}\right) & \frac{{\color{magenta}X_{O_{2}}}}{{\color{cyan}\left(D_{A}\right)_{vap,O_{2}}}} & \frac{{\color{magenta}X_{O_{2}}}}{{\color{cyan}\left(D_{A}\right)_{N_{2},O_{2}}}}\\
\frac{{\color{magenta}X_{vap}}}{{\color{cyan}\left(D_{A}\right)_{O_{2},vap}}} & \left(-\frac{{\color{magenta}X_{O_{2}}}}{{\color{cyan}\left(D_{A}\right)_{vap,O_{2}}}}-\frac{{\color{magenta}X_{N_{2}}}}{{\color{cyan}\left(D_{A}\right)_{vap,N_{2}}}}\right) & \frac{{\color{magenta}X_{vap}}}{{\color{cyan}\left(D_{A}\right)_{N_{2},vap}}}\\
-{\color{magenta}A_{A}}\,\sqrt{{\color{cyan}M_{O_{2}}}} & -{\color{magenta}A_{A}}\,\sqrt{{\color{cyan}M_{vap}}} & -{\color{magenta}A_{A}}\,\sqrt{{\color{cyan}M_{N_{2}}}}
\end{bmatrix}\begin{bmatrix}{\color{magenta}\boldsymbol{N}_{O_{2}}}\\
{\color{magenta}\boldsymbol{N}_{vap}}\\
{\color{magenta}\boldsymbol{N}_{N_{2}}}
\end{bmatrix}=\frac{{\color{cyan}\epsilon}}{{\color{cyan}\tau}^{2}}\begin{bmatrix}{\color{magenta}c_{g}}\,\nabla{\color{red}X_{O_{2}}}\\
{\color{magenta}c_{g}}\,\nabla{\color{red}X_{vap}}\\
\nabla{\color{magenta}P_{g}}
\end{bmatrix}
\]


\[
{\color{magenta}X_{O_{2}}}=\frac{{\color{red}c_{O_{2}}}}{{\color{magenta}c_{g}}}
\]
\[
{\color{magenta}X_{N_{2}}}=\frac{{\color{red}c_{N_{2}}}}{{\color{magenta}c_{g}}}
\]
\[
{\color{magenta}X_{vap}}=\frac{{\color{red}c_{vap}}}{{\color{magenta}c_{g}}}
\]
\[
{\color{magenta}c_{g}}={\color{red}c_{O_{2}}}+{\color{red}c_{N_{2}}}+{\color{red}c_{vap}}
\]
\[
{\color{magenta}P_{g}}={\color{magenta}c_{g}}\,{\color{cyan}R}\,{\color{cyan}T}
\]
\[
\frac{1}{{\color{cyan}\left(D_{A}\right)_{i,j}}}=\frac{1}{{\color{cyan}D_{i,j}}}+\frac{1}{{\color{cyan}D_{i}^{k}}},\,\left(i;j\right)\in\left\{ O_{2};N_{2};vap\right\} ^{2};\, j\neq i
\]
\[
{\color{cyan}D_{i}^{k}}=\frac{2\,{\color{cyan}R_{p}}}{3}\,\sqrt{\frac{8\,{\color{cyan}R}\,{\color{cyan}T}}{\pi\,{\color{cyan}M_{i}}}},\, i\in\left\{ O_{2};N_{2};vap\right\} 
\]
\[
\frac{1}{{\color{magenta}A_{A}}}=\frac{1}{{\color{magenta}A_{C}}}+\frac{1}{{\color{cyan}A_{K}}}
\]
\[
{\color{magenta}A_{C}}=\frac{{\color{cyan}\epsilon}\,{\color{cyan}\mu}}{{\color{magenta}c_{g}}\,{\color{cyan}\tau}^{2}\,{\color{cyan}K}\,{\color{cyan}K_{rg}}\,\sum_{i}\left({\color{magenta}X_{i}}\,\sqrt{{\color{cyan}M_{i}}}\right)}
\]
\[
{\color{cyan}A_{K}}=\frac{3}{4\,{\color{cyan}R_{p}}}\,\sqrt{\frac{\pi\,{\color{cyan}R}\,{\color{cyan}T}}{2}}
\]
\[
{\color{cyan}D_{O_{2},N_{2}}}=\frac{6.43\times10^{-5}\,{\color{cyan}T}^{1.823}}{P_{g}}
\]
\[
{\color{cyan}D_{O_{2},vap}}=\frac{4.26\times10^{-6}\,{\color{cyan}T}^{2.334}}{P_{g}}
\]
\[
{\color{cyan}D_{N_{2},vap}}=\frac{4.45\times10^{-6}\,{\color{cyan}T}^{2.334}}{P_{g}}
\]


La vitesse du gaz est alors donn�e par
\[
{\color{magenta}\boldsymbol{u}_{g}}=\frac{{\color{magenta}\boldsymbol{N}_{O_{2}}}+{\color{magenta}\boldsymbol{N}_{N_{2}}}+{\color{magenta}\boldsymbol{N}_{vap}}}{c_{g}}
\]


\medskip{}


Les constantes sont les suivantes:

\begin{center}
\begin{tabular}{|c|c|c|}
\hline 
Param�tre & Valeur & Unit�\tabularnewline
\hline 
\hline 
$T$ & $353.15$ & K\tabularnewline
\hline 
$R$ & $8.314$ & \tabularnewline
\hline 
$R_{p}$ & $10^{-5}$ & m\tabularnewline
\hline 
$\epsilon$ & $0.7$ & -\tabularnewline
\hline 
$M_{O_{2}}$ & $32\times10^{-3}$ & kg/mol\tabularnewline
\hline 
$M_{N_{2}}$ & $28\times10^{-3}$ & kg/mol\tabularnewline
\hline 
$M_{vap}$ & $18\times10^{-3}$ & kg/mol\tabularnewline
\hline 
$\mu$ & $1.2\times10^{-5}$ & Pa.s\tabularnewline
\hline 
$\tau$ & $3$ & -\tabularnewline
\hline 
$K$ & $6\times10^{-12}$ & m2\tabularnewline
\hline 
$K_{rg}$ & $1$ & -\tabularnewline
\hline 
\end{tabular}
\par\end{center}

\bigskip{}


Les �quations sont �crites sous forme instationnaire mais c'est la
solution stationnaire qui nous int�resse. A noter cependant que sur
des probl�me complet, on peut s'int�resser � des probl�mes instationnaires.


\subsection{Conditions aux limites}

Les flux sont uniform�ment nuls sur les deux fronti�res lat�rales:
\[
\boldsymbol{n}\cdot\boldsymbol{N}_{X}=0,\, X\in\left\{ O_{2};N_{2};vap\right\} 
\]
Les flux sur la fronti�re du bas sont donn�s par
\[
\boldsymbol{n}\cdot\boldsymbol{N}_{O_{2}}=-\frac{i_{0}}{4\, F}
\]
\[
\boldsymbol{n}\cdot\boldsymbol{N}_{N_{2}}=0
\]
\[
\boldsymbol{n}\cdot\boldsymbol{N}_{vap}=+\frac{i_{0}}{2\, F}
\]
avec
\[
i_{0}=10\,000\text{ A/m}{}^{2}
\]
\[
F=96\,500\text{ C/mol}
\]


Les concentrations sont impos�es sur la fronti�re du haut:
\[
c_{O_{2}}=9\text{ mol/m}^{3}
\]
\[
c_{N_{2}}=34\text{ mol/m}^{3}
\]
\[
c_{vap}=8\text{ mol/m}^{3}
\]


\clearpage{}

