
\subsection{Int�r�t du cas-test}

Le principal int�r�t de ce cas est de tester la capacit� � ne pas
r�soudre toutes les �quations dans tous les domaines g�om�triques.
En effet, le transport ionique est r�solu dans la membrane et la couche
active (CA) mais n'est pas r�solu dans la couche de diffusion (GDL),
alors que le transport �lectrique est r�solu dans la couche active
et la couche de diffusion mais pas dans la membrane.

\begin{center}
\input{\orig/transp_ion_elec.pdftex_t}
\par\end{center}


\subsection{G�om�trie}

Les hauteurs des diff�rentes couches sont les suivantes:
\[
h_{mem}=25\,\text{\textmu m}
\]
\[
h_{CA}=10\,\text{\textmu m}
\]
\[
h_{GDL}=200\,\text{\textmu m}
\]


La longueur du domaine est la suivante:
\[
L=2\,\text{mm}
\]



\subsection{�quations}


\subsubsection{Transport �lectrique}

\[
\nabla\cdot\boldsymbol{j}_{e}=S_{e}
\]
\[
\boldsymbol{j}_{e}=-\sigma_{eff}\,\nabla\psi
\]
avec
\[
S_{e}=\begin{cases}
0 & \text{dans la GDL}\\
-i_{e} & \text{dans la CA}
\end{cases}
\]



\subsubsection{Transport ionique}

\[
\nabla\cdot\boldsymbol{j}_{i}=S_{i}
\]
\[
\boldsymbol{j}_{i}=-\kappa_{eff}\,\nabla\phi
\]
avec
\[
S_{i}=\begin{cases}
0 & \text{dans la membrane}\\
+i_{e} & \text{dans la CA}
\end{cases}
\]



\subsubsection{Autres relations et constantes}

\[
i_{e}=i_{0}\,\gamma_{CL}\left[\exp\left(\frac{\alpha nF}{RT}\,\eta\right)-\exp\left(-\frac{\left(1-\alpha\right)nF}{RT}\,\eta\right)\right]
\]
o�
\[
\eta=\left(\psi-\phi\right)-E_{rev}
\]


Les valeurs des param�tres sont donn�es dans le tableau suivant:

\begin{center}
\begin{tabular}{|c|c|c|}
\hline 
Param�tre & Valeur & Unit�\tabularnewline
\hline 
\hline 
$\sigma_{eff}$ & $\begin{cases}
65 & \text{dans la GDL}\\
20 & \text{dans la CA}
\end{cases}$ & S/m\tabularnewline
\hline 
$\kappa_{eff}$ & $\begin{cases}
12 & \text{dans la membrane}\\
1.5 & \text{dans la CA}
\end{cases}$ & S/m\tabularnewline
\hline 
$E_{rev}$ & $1.18$ & V\tabularnewline
\hline 
$\alpha$ & $0.5$ & -\tabularnewline
\hline 
$n$ & $2$ & -\tabularnewline
\hline 
$R$ & $8.314$ & \tabularnewline
\hline 
$T$ & $353.15$ & K\tabularnewline
\hline 
$i_{0}$ & $10^{-5}$ & A/m$^{2}$\tabularnewline
\hline 
$F$ & $96500$ & C/mol\tabularnewline
\hline 
$\gamma_{CL}$ & $2.5\times10^{7}$ & m$^{-1}$\tabularnewline
\hline 
\end{tabular}
\par\end{center}


\subsection{Conditions aux limites}


\subsubsection{Transport �lectrique}

\[
\psi=0.7
\]
sur la fronti�re sup�rieure de la GDL et
\[
\boldsymbol{n}\cdot\boldsymbol{j}_{e}=0
\]
ailleurs.


\subsubsection{Transport ionique}

\[
\phi=0
\]
sur la fronti�re inf�rieure de la membrane et
\[
\boldsymbol{n}\cdot\boldsymbol{j}_{i}=0
\]
ailleurs.


\subsection{Cas-tests suppl�mentaires}


\subsubsection{Collecte de courant localis�e}

Pour s'approcher plus du cas d'application, la condition
\[
\psi=0.7
\]
n'est impos�e que sur le tiers central de la fronti�re sup�rieure
de la GDL, la condition
\[
\boldsymbol{n}\cdot\boldsymbol{j}_{e}=0
\]
sur les deux autres tiers ainsi que sur les autres fronti�res.


\subsubsection{Champ de conductivit� �lectrique variable}

Pour prendre en compte l'un des effets de l'appui de la dent collectrice
de courant, les champs $\sigma_{eff}$ et $\kappa_{eff}$ sont des
fonctions de l'espace qui sont donn�es.

\clearpage{}

